\documentclass[a4paper, 12pt, headsepline]{scrartcl}
    % General document formatting
    \usepackage[margin=0.7in]{geometry}
    \usepackage[parfill]{parskip}
    \usepackage[utf8]{inputenc}
    \usepackage[headsepline,footsepline]{scrpage2}
    

% Header and Footer Formatting-----------------------------------------------------------------------
%   Formatting of Header and Footer
% ---------------------------------------------------------------------------------------------------
\pagestyle{scrheadings}
\clearscrheadfoot
\ohead{\headmark}
\ofoot{\pagemark}

% Title Page Information-----------------------------------------------------------------------------
%   These are the informations printed on the title page of the article
% ---------------------------------------------------------------------------------------------------

\title{Project trckr}
\subtitle{Technical Article}
\date{May \\ 2018}
\author{
Ankeshian, Gabriel\\
\texttt{ankesgab@students.zhaw.ch}
\and
Balidis, Dimitri\\
\texttt{baliddim@students.zhaw.ch}
\and
Christen, Luca\\
\texttt{chrisluc@students.zhaw.ch}
\and
Jossi, Savino\\
\texttt{jossisav@students.zhaw.ch}
\and
Milenkovic, Daniel\\
\texttt{milendan@students.zhaw.ch}
\and
Nominato, Angelica Helena Moreira Alves\\
\texttt{moreiane@students.zhaw.ch}
\and
Pacassi, David\\
\texttt{pacasdav@students.zhaw.ch}}

\begin{document}
\maketitle
\pagebreak

% Abstract-----------------------------------------------------------------------------
%   This is the abstract of the technical article
% -------------------------------------------------------------------------------------
\begin{abstract}
The main goal of the present article is to describe the idea, goals and main functionalities of the web based app trckr. The app is being used
to track the time on single tasks of projects assigned to a person.\\
The app consists of a back- and frontend. The backend is written in Python with the help of the framework Django, the frontend with the Javascript
framework Vue.js.

\end{abstract}

\pagebreak

% Table of Contents-----------------------------------------------------------------------------
%   This is the table of contents
% -------------------------------------------------------------------------------------------

\tableofcontents

\pagebreak

% Introduction-----------------------------------------------------------------------------
%   These are the informations printed on the title page of the article
% ------------------------------------------------------------------------------------------
\section{Introduction}
Introduction...

\subsection{Initial Situation and Objectives}

\subsubsection{Initial Situation}

\subsubsection{Objectives}

\subsection{Main Features}

\subsection{Technolgies}


\subsubsection{Django}


\subsubsection{PostgreSQL}


\subsubsection{Vue.js}



% Results-----------------------------------------------------------------------------
%   These are the informations printed on the title page of the article
% -------------------------------------------------------------------------------------------
\section{Results}
Results...

% Outlook-----------------------------------------------------------------------------
%   These are the informations printed on the title page of the article
% -------------------------------------------------------------------------------------------
\section{Outlook}
Outlok...

% Conclusion-----------------------------------------------------------------------------
%   These are the informations printed on the title page of the article
% -------------------------------------------------------------------------------------------
\section{Conclusion}
Conclusion...

% Bibliography-----------------------------------------------------------------------------
%   These are the informations printed on the title page of the article
% -------------------------------------------------------------------------------------------
\section{Bibliography}
Bibliography...
\end{document}