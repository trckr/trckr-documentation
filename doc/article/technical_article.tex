\documentclass[a4paper, 12pt, headsepline]{scrartcl}
    % General document formatting
    %\usepackage[margin=0.7in]{geometry}
    %\usepackage[parfill]{parskip}
    \usepackage[utf8]{inputenc}
    %\usepackage[headsepline,footsepline]{scrpage2}
    \usepackage[onehalfspacing]{setspace}
    \usepackage{times}
    \usepackage{graphicx}
    \graphicspath{ {images/} }


% Header and Footer Formatting-----------------------------------------------------------------------
%   Formatting of Header and Footer
% ---------------------------------------------------------------------------------------------------
%\pagestyle{scrheadings}
%\clearscrheadfoot
%\ohead{\headmark}
%\ofoot{\pagemark}

% TODO: make itemize neater
% TODO: figure links

% Center all images
\makeatletter
\g@addto@macro\@floatboxreset\centering
\makeatother

% Title Page Information-----------------------------------------------------------------------------
%   These are the informations printed on the title page of the article
% ---------------------------------------------------------------------------------------------------

\title{Project trckr}
\subtitle{Technical Article}
\date{May 2018}
% TODO: authors more compact? 
\author{
Ankeshian, Gabriel\\
\texttt{ankesgab@students.zhaw.ch}
\and
Balidis, Dimitri\\
\texttt{baliddim@students.zhaw.ch}
\and
Christen, Luca\\
\texttt{chrisluc@students.zhaw.ch}
\and
Jossi, Savino\\
\texttt{jossisav@students.zhaw.ch}
\and
Milenkovic, Daniel\\
\texttt{milendan@students.zhaw.ch}
\and
Nominato, Angelica Helena Moreira Alves\\
\texttt{moreiane@students.zhaw.ch}
\and
Pacassi, David\\
\texttt{pacasdav@students.zhaw.ch}}

\begin{document}
\maketitle
\pagebreak

% Proofreading Checklist
% * avoid passive voice
% * succinct sentences
% * use active voice
% * did I say don't use passive voice?

% Abstract----------------------------------------------------------------------
%   This is the abstract of the technical article
% ------------------------------------------------------------------------------
\begin{abstract}
  The main goal of the present article is to describe the idea, goals and main
  functionalities of the web based application trckr. We developed trckr for
  % Since we want trckr to be all lowercase, we avoid using it at the start of
  % the sentence
  everyone who works on a project and needs an intuitive and simple web tool to
  accurately track their time spent on different tasks.
  % Hint: Empty line makes a paragraph which is semantically different from the
  % forced \\ newline (and much neater)

  The backend is written in Python with the help of the web application
  framework Django and the frontend with the Javascript UI framework Vue.js.
  Both technologies were new to most team members, but have proven themselves
  effective and learning them were ultimately a benefit the outcome of the
  project.

  In order to compete with similar tools and web services, trckr focuses on
  performance and usability. To distinguish trckr from the competition, many
  features are planned to manage projects and tasks in a user-friendly manner.
  This will allow the user to leverage trckr to handle the ever increasing
  complexity in project management and task tracking found in large companies.
  Despite being targeted at large companies, trckr will remain open source and
  anybody can contribute, covering cases we might have never dreamed of.
\end{abstract}

\clearpage

\tableofcontents

\clearpage

\section{Introduction}
Time and task tracking are important activities in many businesses to gain
insights on the productivity of a team, requiring appropriate tools allowing
easier and more accurate time tracking. Many processes and methods have been
developed in the past to cover this need. Unfortunately most of them address
just a certain need or have been fine tuned to a specific company or team. This
necessarily leads to a loss of experience that could have been leveraged by
other teams but is also not generic enough to adapt to different processes,
causing other companies to inappropriately adapt to the tool instead of the tool
adapting to the company.

\subsection{Objectives}
The goal with trckr is to develop and distribute a time tracking web application
that is easy to understand and use. The most important requirement is to
require just a few steps to track all the required information. Only this will
keep the user engaged and raise the accuracy of the provided data.

\subsection{Main Features}
The user is able to:
\begin{itemize}
    \item register and login to trckr
    \item create and edit projects
    \item create, track and edit tasks
    \item visit trckr on any device
\end{itemize}

\section{Technologies}
\subsection{Django}
Django is an open source web framework written in Python. It encourages fast,
clean and simple development of web applications. One of the main advantages is
it's fast setup, enabling the developer to create applications swiftly through
it's Model-View-Presenter scheme. Django also comes with support for various
databases. Many users compare Django to Ruby On Rails but written in Python.
Django also follows the DRY principle (Don't Repeat Yourself).

\subsection{PostgreSQL}
A PostgreSQL database stores all the data for trckr. PostgreSQL is fully
supported by Django and configuration and management is minimal, because the
database access is handled by Django.

\subsection{Vue.js}
Vue.js is a progressive framework for building user interfaces.\cite{vuejs}
Vue.js is used for trckr mainly for its simplicity and the rather shallow
learning curve it provides to unexperienced developers. The features that Vue.js
provides allow the creation of data structures that can easily be displayed in
on a website. This and the ability to easily make calls to the backend make it a
good fit for trckr.

\section{Results}
The architecture was kept fairly simple, as illustrated in figure
\ref{fig:architecture}, using Django with PostgreSQL in the backend and a
frontend based around Vue.js.

\begin{figure}[h]
    \includegraphics[width=8cm]{architecture}
    \caption{Architecture of trckr}
    \label{fig:architecture}
\end{figure}

\subsection{API}
The backend of the trckr application implements a RESTful API using the Django
REST framework. The API provides basic CRUD operations for all the entities
available in the database. There are five main endpoints to retrieve and save
data on the server: authentication, user, projects, tasks and time entries. Except
for the authentication and user endpoints, all endpoints need an authentication
token to be accessed.

\subsubsection{Endpoints}
\begin{description}
\item[authentication] allows users to retrieve an authentication token from the
  server to access the other parts of the API. Via this endpoint, one can also
  invalidate the token.
\item[user] is only used to create new user accounts.
\item[projects] lets user create, read, update and delete projects as well as
  display all the tasks associated with a project.
\item[task] used to create, read and update tasks for a given project. There
  also exits a way to list all relevant time entries for a task.
\item[time entries] also used for create, read, update and delete operations
\end{description}

Each object of an entity has a unique ID. This ID can be used to retrieve
information for that specific object by providing it in the URL when calling the
server. This is necessary when updating an object via a POST request.

\subsection{User Interface}
% SJ: I removed details about the implementation (receive a token) and rewrote
% to reflect the UX

% TODO: Check figures and references for mismatches

When opening the trckr website the first time you are presented with the login
screen as shown in figure \ref{fig:trckr-login}. For users who have not yet
registered an account, this can be done here by entering some basic information
like username, password, email address and first and last name (Figure
\ref{fig:trckr-register}).

\begin{figure}[h]
    \includegraphics[width=0.8\textwidth]{trckr-login}
    \caption{The trckr login page.}
    \label{fig:trckr-login}
\end{figure}

\begin{figure}[h]
    \includegraphics[width=0.8\textwidth]{trckr-register}
    \caption{The trckr registration page.}
    \label{fig:trckr-register}
\end{figure}

Once the registration process has been completed, the
user can log in at the login page. Once the user is logged in, the navigation
bar at the top of the interface will contain links to the dashboard, the
projects page and the time entries page as well as a logout button. The dashboard
displays graphs... TODO

The projects page shows a table of all projects that the currently logged in
user is a part of, seen in figure \ref{fig:trckr-project-page}. There is a
search box above the projects list that allows a user to filter for a project or
a group of projects containing the given keywords. To create a new project the
user will have to navigate to the projects page and click on the "Create
project" link which will open the project creation form. % please add figure
The form asks for the name of the project and optionally allows the user to enter an
description.

Each project can be viewed in more detail when the project inside the table is
clicked, as shown in figure \ref{fig:trckr-project-page}. The project page
contains the name, the description and a table of all the tasks in the selected
project.

\begin{figure}[h]
    \includegraphics[width=\textwidth]{trckr-projects-table}
    \caption{The projects page.}
    \label{fig:trckr-projects-table}
\end{figure}

\begin{figure}[h]
    \includegraphics[width=\textwidth]{trckr-project-page}
    \caption{The projects page with the table of all projects.}
    \label{fig:trckr-project-page}
\end{figure}

Clicking on a task will display it's details like the name and description of
the task. On the "Time Entries" page a user can create a time entry for a
specific task of a project with the entry form shown in figure
\ref{fig:trckr-create-time-entry}. The form asks the user to first choose a
project and then a task of the project for which the time entry should be
created.

\begin{figure}[h]
    \includegraphics[width=\textwidth]{trckr-project-page}
    \caption{The form to add a time entry to a task.}
    \label{fig:trckr-create-time-entry}
\end{figure}

All time entries will be displayed on the time entry page.

\section{Outlook}
Outlook...

\section{Conclusion}
Conclusion...

\section{Bibliography}
Bibliography...
\begin{thebibliography}{9}

    \bibitem{vuejs}
    https://vuejs.org/ ; as 10.05.2018


\end{thebibliography}

\listoffigures

\end{document}
