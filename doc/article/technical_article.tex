\documentclass[a4paper, 12pt, headsepline]{scrartcl}
    % General document formatting
    %\usepackage[margin=0.7in]{geometry}
    %\usepackage[parfill]{parskip}
    \usepackage[utf8]{inputenc}
    %\usepackage[headsepline,footsepline]{scrpage2}
    \usepackage[onehalfspacing]{setspace}
    \usepackage{times}
    \usepackage{graphicx}
    \graphicspath{ {images/} }
    

% Header and Footer Formatting-----------------------------------------------------------------------
%   Formatting of Header and Footer
% ---------------------------------------------------------------------------------------------------
%\pagestyle{scrheadings}
%\clearscrheadfoot
%\ohead{\headmark}
%\ofoot{\pagemark}

% Title Page Information-----------------------------------------------------------------------------
%   These are the informations printed on the title page of the article
% ---------------------------------------------------------------------------------------------------

\title{Project trckr}
\subtitle{Technical Article}
\date{May 2018}
\author{
Ankeshian, Gabriel\\
\texttt{ankesgab@students.zhaw.ch}
\and
Balidis, Dimitri\\
\texttt{baliddim@students.zhaw.ch}
\and
Christen, Luca\\
\texttt{chrisluc@students.zhaw.ch}
\and
Jossi, Savino\\
\texttt{jossisav@students.zhaw.ch}
\and
Milenkovic, Daniel\\
\texttt{milendan@students.zhaw.ch}
\and
Nominato, Angelica Helena Moreira Alves\\
\texttt{moreiane@students.zhaw.ch}
\and
Pacassi, David\\
\texttt{pacasdav@students.zhaw.ch}}

\begin{document}
\maketitle
\pagebreak

% Abstract-----------------------------------------------------------------------------
%   This is the abstract of the technical article
% -------------------------------------------------------------------------------------
\begin{abstract}
The main goal of the present article is to describe the idea, goals and main functionalities of the web based app trckr. Trckr is for everyone,
who works on a project and wants an intuitive and simple web tool with easy to learn handling.\\
To have an easy development, the backend is written in Python with the help of the framework Django, the frontend with the Javascript
framework Vue.js. Users are able to create and edit projects after a successful registration. Time tracking starts with single tasks
a user creates.\\
In order to compete with other tools and web services, trckr has big advantages in performance and usability. To extend the reach of trckr, more
functions are planned to display projects and tasks in a userfriendly manner. These plannded features will give trckr a great advantage towards ever 
increasing complexity in project managment and task tracking, this would be especially useful for larger companies. Additionally we provide trckr
as an open source solution and everybody can contribute features that could be useful for the greater userbase.

\end{abstract}

\pagebreak

% Table of Contents-----------------------------------------------------------------------------
%   This is the table of contents
% -------------------------------------------------------------------------------------------

\tableofcontents

\pagebreak

% Introduction-----------------------------------------------------------------------------
%   These are the informations printed on the title page of the article
% ------------------------------------------------------------------------------------------
\section{Introduction}
Time tracking is an important process for daily business to have insight on the productivity of a team, this lead to ever improving processes and tools
that allow for easier time tracking, no matter which branch. This also lead to multiple methods and tools being developed and enhanced in parallel,
many methods are not very helpful for a certain branch because they might give enough insight or have too many features that are not going to be used.
Yet these methods might be good for another branch, this tells us, that diversity is in no way an issue and tools and methods are adapting to teams,
and not the other way around.

\subsection{Objectives}
The goal with project trckr is to develop and distribute a time tracking webapp, that is very easy to understand and use. It is highly important to
have the least amount of steps possible, for a user to track his time on single tasks.

\subsection{Main Features}
The user is able to:
\begin{itemize}
    \item register and login to trckr
    \item create and edit projects
    \item create, track and edit tasks
    \item visit trckr also on a mobile browser
\end{itemize}


\section{Technolgies}


\subsection{Django}
Django is an open source web framework written in Python. It is very useful for fast, clean and simple development of web applications.\\
One of the main advantages is its fast setup, therefore you can start developing your application very quick.
Django comes with support of various database integrations.

\subsection{PostgreSQL}
Trckr runs with a PostgreSQL database in the backend.

\subsection{Vue.js}
Vue.js is a progressive framework for building user interfaces. Trckr is developed with Vue.js mainly for its simplicity and the rather shallow learning curve it provides to unexpirienced developers. 
The features that Vue.js provides, allow the creation of data structures that can easily be displayed in the HTML of a page. This and the ability to easily make calls to the backend makes it a perfect allround framework for trckr.


% Results-----------------------------------------------------------------------------
%   These are the informations printed on the title page of the article
% -------------------------------------------------------------------------------------------
\section{Results}
As you can see in the figure \ref{fig:architecture}, we have built a pretty simple architecture,
including Django with PostgreSQL in the backend and a frontend based on Node.
 
\begin{figure}[h]
    \includegraphics[width=8cm]{architecture}
    \caption{Architecture of trckr}
    \label{fig:architecture}
\end{figure}

\subsection{APIs}


\subsection{User Interface}

% Outlook-----------------------------------------------------------------------------
%   These are the informations printed on the title page of the article
% -------------------------------------------------------------------------------------------
\section{Outlook}
Outlook...

% Conclusion-----------------------------------------------------------------------------
%   These are the informations printed on the title page of the article
% -------------------------------------------------------------------------------------------
\section{Conclusion}
Conclusion...

% Bibliography-----------------------------------------------------------------------------
%   These are the informations printed on the title page of the article
% -------------------------------------------------------------------------------------------
\section{Bibliography}
Bibliography...

\listoffigures

\end{document}