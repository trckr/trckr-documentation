\documentclass[a4paper, 12pt, headsepline]{scrartcl}
    % General document formatting
    %\usepackage[margin=0.7in]{geometry}
    %\usepackage[parfill]{parskip}
    \usepackage[utf8]{inputenc}
    %\usepackage[headsepline,footsepline]{scrpage2}
    \usepackage[onehalfspacing]{setspace}
    

% Header and Footer Formatting-----------------------------------------------------------------------
%   Formatting of Header and Footer
% ---------------------------------------------------------------------------------------------------
\pagestyle{scrheadings}
\clearscrheadfoot
\ohead{\headmark}
\ofoot{\pagemark}

% Title Page Information-----------------------------------------------------------------------------
%   These are the informations printed on the title page of the article
% ---------------------------------------------------------------------------------------------------

\title{Project trckr}
\subtitle{Technical Article}
\date{May \\ 2018}
\author{
Ankeshian, Gabriel\\
\texttt{ankesgab@students.zhaw.ch}
\and
Balidis, Dimitri\\
\texttt{baliddim@students.zhaw.ch}
\and
Christen, Luca\\
\texttt{chrisluc@students.zhaw.ch}
\and
Jossi, Savino\\
\texttt{jossisav@students.zhaw.ch}
\and
Milenkovic, Daniel\\
\texttt{milendan@students.zhaw.ch}
\and
Nominato, Angelica Helena Moreira Alves\\
\texttt{moreiane@students.zhaw.ch}
\and
Pacassi, David\\
\texttt{pacasdav@students.zhaw.ch}}

\begin{document}
\maketitle
\pagebreak

% Abstract-----------------------------------------------------------------------------
%   This is the abstract of the technical article
% -------------------------------------------------------------------------------------
\begin{abstract}
The main goal of the present article is to describe the idea, goals and main functionalities of the web based app trckr. Trckr is for everyone,
who works on a project and wants a very simple web tool with intiutive handling.\\
To have an easy development, the backend is written in Python with the help of the framework Django, the frontend with the Javascript
framework Vue.js. Users are able to create and edit projects after a successful registration. Time tracking starts with single tasks
a user creates.\\
In order to compete with other tools and web services, trckr has big advantages in performance and usability. To extend the reach of trckr, more
functions are planned to display projects and tasks in a very userful way. Furthermore the functionality for better implementation in big
companies would be a big step, to reach also bigger complexities of project management and task tracking.

\end{abstract}

\pagebreak

% Table of Contents-----------------------------------------------------------------------------
%   This is the table of contents
% -------------------------------------------------------------------------------------------

\tableofcontents

\pagebreak

% Introduction-----------------------------------------------------------------------------
%   These are the informations printed on the title page of the article
% ------------------------------------------------------------------------------------------
\section{Introduction}
Time tracking has always been a really important thing in daily tasks and projects, and so tools and methods for time tracking have evolved.
There are dozens of methods, that are teached in project management and self improvement classes. Some of them are overkills and some of them are
very useful, depending on a person, task and project.

\subsection{Objectives}
The goal with project trckr is to develop and distribute a time tracking webapp, that is very easy to understand and use. It is very important to
have a very small amount of steps for a user to track his time on single tasks.

\subsection{Main Features}
The user is able to:
\begin{itemize}
    \item register and login to trckr
    \item create and edit projects
    \item create, track and edit tasks
    \item visit trckr also on a mobile browser
\end{itemize}


\section{Technolgies}


\subsection{Django}
Django is an open source web framework written in Python. It is very useful for fast, clean and simple development of web applications.\\
One of the main advantages is its fast setup, therefore you can start developing your application very quick.
Django comes with support of various database integrations.

\subsection{PostgreSQL}
Trckr runs with a PostgreSQL database in the backend.

\subsection{Vue.js}
Vue is a progressive framework for building user interfaces. Trckr is developed with vue.js because of it's
very easy to learn modules. Also it'ts very handy when it comes to developing user interfaces.



% Results-----------------------------------------------------------------------------
%   These are the informations printed on the title page of the article
% -------------------------------------------------------------------------------------------
\section{Results}
Results...

\subsection{APIs}


\subsection{User Interface}

% Outlook-----------------------------------------------------------------------------
%   These are the informations printed on the title page of the article
% -------------------------------------------------------------------------------------------
\section{Outlook}
Outlok...

% Conclusion-----------------------------------------------------------------------------
%   These are the informations printed on the title page of the article
% -------------------------------------------------------------------------------------------
\section{Conclusion}
Conclusion...

% Bibliography-----------------------------------------------------------------------------
%   These are the informations printed on the title page of the article
% -------------------------------------------------------------------------------------------
\section{Bibliography}
Bibliography...
\end{document}